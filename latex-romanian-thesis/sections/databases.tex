\chapter{Experiments}
Given that at the core of the application are learning algorithms, the system reliability is determined by the databases from which train and test data is used. Although for the face embedding part we were not able to train the neural network ourselves given the limited computational power at our disposal, for the face validation part, the public domain databases offered us a large choice, the most commonly used ones are described in the following sections with the corresponding experiments. In order to learn the face distance threshold, we used another database, MS-Celeb-1M \cite{guo2016msceleb}, but given it's purpose for training neural networks, we only used a small sample which was enough to give us sufficient information about the optimum value of the parameter.
\section{Face spoof databases}
In this section we analyze the accuracy of our LBP-implementation of a face liveness detection method based on the most commonly used anti-spoof face databases available in public domain. We present our intra-database experiments as well as cross-database ones, the latter one being considered better in modeling a real use case.
\subsection{CASIA face antispoofing database }
CASIA database for face antispoofing was created by the Center of Biometrics and Security Research (CBRS) and is composed by 600 videos divided in three categories: low, normal and high quality of 50 asian subjects, 12 videos for each. Out of the 12 videos, 3 are genuine and 9 are fake. There are implemented three face attacks which include: warped photo attack in which the printed photo is bended over the subjects face, cut photo where the eyes on the printed photo are cropped and then position on in front of the subject's face so that the blinking is made possible and video attack replayed on an IPAD.

\subsection{MSU USSA}
\subsection{MSU MFSD}
\subsection{IDIAP Replay-Attack database}
\subsection{Interdatabase testing}
\section{Face distance threshold}
In order to determine the best value for the face distance threshold which, as described in section \ref{face_recognition} represents the maximum distance between the embeddings of two faces at which we can still consider them as belonging to the same identity, we used a sample from the MS-Celeb-1M database of celebrities faces put together by Microsoft in order to create a benchmark task to recognize one million celebrities from their face images by using as training data all the face images of the individual that are available on the internet. Here we did three experiments in determining the threshold value based on the proposed approach the results of which can be seen in figure \ref{label_here_mofo}. 

\textbf{The first experiment} consisted of randomly selecting \textbf{101} identities and for each one sampling in a random way about \textbf{26} faces. This produced a number of \textbf{2.626} total face images from which we computed \textbf{6.895.876} pairs out of which \textbf{68.276} had the same identity and \textbf{6.827.600} had different identities. This produces a ratio of \textbf{0.01} same identity to different identities pairs.

\textbf{The second experiment} was conducted in the same way as the first one but with different number of subjects, specifically \textbf{296} and about \textbf{26} faces for every identity. This produced a number of \textbf{7.696} total face images from which we computed \textbf{59.228.416} pairs out of which \textbf{200.096} had the same identity and \textbf{59.028.320} had different identities. This produces a ratio of \textbf{0.03} same identity to different identities pairs.

\textbf{The third experiment} differed from the first two in that we selected a smaller number of identities but we increased the number of faces per identity. This way we used the online available MS-Celeb-1M sample data containing \textbf{12} identities with approximately \textbf{100} faces per identity. Using this dataset, we had about \textbf{1.200} faces combined together in \textbf{1.440.000} pairs out of which \textbf{120.000} pairs with faces of the same identity and \textbf{1.320.000} with different identities, result in a ratio of \textbf{0.09} same identity to different identities pairs.

As a result of the three experiments, it can be seen that the embedding function that produces the feature vector used to compute the distances has a stable face distance threshold at around \textbf{1.1} which is invariant to number of identities and face images.